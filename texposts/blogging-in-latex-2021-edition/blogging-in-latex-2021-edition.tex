\documentclass{article}
\usepackage{hyperref}
% \usepackage{markdown}
\title{Blogging in LaTeX, 2021 edition}
\author{Michal Hoftich}
\begin{document}
\maketitle

% \begin{markdown*}{hybrid=true}


This is the first in a series of posts about blogging with \LaTeX\ using
\href{https://tug.org/tex4ht/}{\TeX4ht}. I set it up as a showcase of
\TeX4ht features for the \href{https://tug.org/tug2021/}{TUG 2021} conference.
However, I didn't present it in the end because I ran out of the allocated time
for the presentation and didn't finish all the posts on time.

At present, the articles are still drafts. I plan to update them over time,
but my progress, as usual with my writing, may be slow.

\tableofcontents

\section{Why would you do such a crazy thing?}

I like writing my documents in \LaTeX, and I want to provide an alternative
to other markup languages. The nice thing about \LaTeX\ is that it gives you 
a great flexibility in the way how to write your documents. 

The nice thing about \LaTeX\ is that it gives you great flexibility in how you
write your documents. You can define your own environments and commands, which
allows you to structure content more effectively and reduce repetition. This
feature is especially useful when writing large documents, such as books or
theses, where consistency is crucial, but it can be useful even for shorter articles. 


Moreover, \LaTeX\ provides access to a vast collection of packages that extend
its capabilities. For example, you can create professional-looking diagrams
using packages like \texttt{TikZ} or \texttt{PGFPlots}, write complex
mathematical formulas with \texttt{Amsmath}, or handle bibliographies
efficiently using \texttt{BibLaTeX}.

\section{Overview of the setup}

Contrary to some claims, compiling \LaTeX\ to HTML
isn't difficult—you just need to take some precautions. I am the author of 

\section{Overview of the setup}

\begin{enumerate}
\item Use \href{/testblog/2021/07/30/how-to-blog-with-tex4ht.html}
{\TeX4ht to produce files suitable for static site generators}.
\item Use a static site generator, such as \href{https://jekyllrb.com/}{Jekyll}
or \href{https://gohugo.io/}{Hugo}, to generate a website.
\item Use GitHub Actions to automatically compile \LaTeX\ files pushed
to the repository and rebuild the website.
\end{enumerate}

In the following posts, I will explain how to set up an automatic
blog publishing system using Jekyll, GitHub Pages, and GitHub Actions.

The code for this blog is available at
\url{https://github.com/michal-h21/testblog}.

Notable files and directories include:

\begin{itemize}
  \item \href{https://github.com/michal-h21/testblog/blob/main/.make4ht}{Build file for make4ht.}
  \item \href{https://github.com/michal-h21/testblog/tree/main/texposts}{Root directory for \LaTeX\ documents.}
  \item \href{https://github.com/michal-h21/testblog/blob/main/texposts/rebuild.sh}{Script to force rebuild only for changed files},
    as we don't want to compile all \LaTeX\ files on every blog update.
  \item \href{https://github.com/michal-h21/testblog/blob/main/.github/workflows/main.yml}{GitHub Actions workflow file.}
\end{itemize}

By following these steps, you can create a fully automated blogging workflow with \LaTeX\ and \TeX4ht. 
This approach allows you to focus on writing while the technical aspects of publishing are handled seamlessly. 
At least i
In the upcoming posts, I will cover additional details, such as customizing the output, managing dependencies, 
and troubleshooting common issues. Stay tuned for more insights on making the most of \LaTeX\ for blogging!


\end{document}
