\documentclass{article}
\usepackage{hyperref}
% \usepackage{markdown}
\title{Blogging in LaTeX, 2025 edition}
\author{Michal Hoftich}
\begin{document}
\maketitle

% \begin{markdown*}{hybrid=true}

This is the first post in a series about blogging with \LaTeX\ using
\href{https://tug.org/tex4ht/}{\TeX4ht}. I originally set it up as a showcase
of \TeX4ht’s features for the \href{https://tug.org/tug2021/}{TUG 2021}
conference. Unfortunately, I didn’t get to present it in the end—I ran out of
time during my allocated slot, and I also didn’t manage to finish all the posts
I had planned by the deadline. Fast forward to February 2025, and I thought,
why not finally complete this series? Better late than never, right?


\tableofcontents

\section{Why would you do such a crazy thing?}


I really enjoy writing my documents in \LaTeX, and I want to offer a solid
alternative to other markup languages like Markdown or HTML. The beauty of
\LaTeX\ lies in its incredible flexibility—it lets you structure your documents
exactly how you want, without being tied down by the limitations of more rigid
systems. Whether you're writing a quick report or a full-blown book, \LaTeX\
gives you the tools to make it look polished and professional.

Of course, \LaTeX\ isn’t for everyone, and that’s totally fine! Some people
prefer the instant visual feedback you get with WYSIWYG editors like Word or
Google Docs, and there’s absolutely nothing wrong with that. Others might like
the simplicity of plain text but find \LaTeX’s syntax too complicated or even a
bit ugly—and hey, that’s fair too. Markdown and other lightweight markup
languages are great alternatives, and there’s no shame in using them.

Let’s be honest, \LaTeX\ can be pretty complex, and a lot of people won’t even
scratch the surface of what it can do. And that’s okay—not everyone needs that
level of control or customization. Personally, though, I’ve never been a fan of
Word or Markdown. Word feels too restrictive for my taste, and Markdown, while
simple, just doesn’t give me the power I need. That’s why I’ve turned to
\LaTeX\ as my go-to tool. It’s my way of creating an alternative for myself and
others who feel the same way—people who want more flexibility and precision
without being tied down by the limitations of other tools.

One of the coolest things about \LaTeX\ is how customizable it is. You can
create your own environments and commands, which is a game-changer when it
comes to organizing your content. This is super handy for big projects like
theses or books, where you need everything to be consistent, but it’s also
great for smaller stuff like articles or essays. Plus, once you’ve set up your
custom commands, you can reuse them across different documents, saving you a
ton of time.

Another reason I love \LaTeX\ is the sheer number of packages available. These
packages let you do pretty much anything you can think of. Need to draw
diagrams? Check out \texttt{TikZ} or \texttt{PGFPlots}. Writing a math-heavy
paper? \texttt{Amsmath} has got your back. Managing references?
\texttt{BibLaTeX} makes it a breeze. The possibilities are endless, and it’s
one of the reasons \LaTeX\ is so powerful.

Now, some people say that converting \LaTeX\ to HTML is a pain, but honestly,
it’s not that hard if you know what you’re doing. I’m the author of
\href{https://tug.org/tex4ht/}{\TeX4ht}, a tool that makes this process a lot
smoother. It’s a great way to make your content more accessible online without losing the
precision and beauty of \LaTeX.

\section{Overview of the setup}

In the upcoming posts, we’ll dive into how \TeX4ht works and how to configure
it to generate HTML files that play nicely with static site generators. After
that, I’ll share how I used Jekyll to build this series of posts. 
Finally, we’ll wrap things up by exploring how to use GitHub Actions to
automatically rebuild your website whenever you update your \LaTeX\ documents.

Here is an outline of the steps we’ll cover:

\begin{enumerate}
\item Use \href{/testblog/2021/07/30/how-to-blog-with-tex4ht.html}
{\TeX4ht to produce files suitable for static site generators}.
\item Use a static site generator, such as \href{https://jekyllrb.com/}{Jekyll}
or \href{https://gohugo.io/}{Hugo}, to generate a website.
\item Use GitHub Actions to automatically compile \LaTeX\ files pushed
to the repository and rebuild the website.
\end{enumerate}


If you’re curious to check out the source code for this blog, you can find it at this link:
\url{https://github.com/michal-h21/testblog}.

Notable files and directories include:

\begin{itemize}
  \item \href{https://github.com/michal-h21/testblog/blob/main/.make4ht}{Build file for make4ht.}
  \item \href{https://github.com/michal-h21/testblog/tree/main/texposts}{Root directory for \LaTeX\ documents.}
  \item \href{https://github.com/michal-h21/testblog/blob/main/texposts/rebuild.sh}{Script to force rebuild only for changed files},
    as we don't want to compile all \LaTeX\ files on every blog update.
  \item \href{https://github.com/michal-h21/testblog/blob/main/.github/workflows/main.yml}{GitHub Actions workflow file.}
\end{itemize}


\end{document}
