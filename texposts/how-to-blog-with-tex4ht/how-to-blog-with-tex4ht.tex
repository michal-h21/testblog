\documentclass{article}
\usepackage{hyperref}
\title{How to Blog With TeX4ht}
\author{Michal Hoftich}
\def\makeht{\texttt{make4ht}}
\begin{document}
\maketitle

This post is a part of series on how to set up
\href{https://tug.org/tex4ht/}{TeX4ht}, \LaTeX\ to XML converter
for use with Static Site Generators. We will discuss
how to configure it to produce suitable HTML in this article.

\tableofcontents

\section{Static site extension for make4ht}

Conversion process used by \TeX4ht is quite complex. It needs to 
compile \LaTeX\ file to a DVI file with special instructions 
inserted by \texttt{tex4ht.sty} package. This DVI file is then 
processed by \texttt{tex4ht} command, that produces HTML or XML files,
and instructions for the last command, \texttt{t4ht}, which produces
CSS files and pictures. 


Traditionally, this process was handled 
by the \texttt{htlatex} script, but it had many weaknesses, so 
the currently recommended build tool is \makeht. 
You can find some details about \texttt{htlatex} and \makeht\ 
differences in the 
\href{https://www.kodymirus.cz/make4ht/make4ht-doc.html#difference-of-texttt-makeht}
{\makeht\ documentation}.

Among features provided by \makeht\ are Lua build files, post-processing filters, and extensions.
We can use these features to transform HTML files produced by \TeX4ht to form required by 
static site generators.

Filters can clean-up the generated files, and fix some common issues that 
are hard to fix on the \TeX\ level. They can be applyed either from 
Lua build files, or using \makeht\ extensions. 

\makeht\ provides an extension that aims at support for static site generators. We can show the usage
on a simple example:

\begin{verbatim}

\documentclass{article}
\begin{document}
\title{Hello world test}
\author{Michal}
\maketitle

This is my test post.
\end{document}
\end{verbatim}

You can use a following command to generate file suitable for static site generators:

\begin{verbatim}
make4ht -f html5+staticsite filename.tex
\end{verbatim}

By default, \texttt{staticsite} extension produces file named as \verb|YYYY-MM-DD-<filename>|,
this example can be named as \texttt{2021-07-25-filename.html}. It is not 
ordinary HTML file, but it contains YAML header with document metadata:

\begin{verbatim}
---
meta:
- charset: 'utf-8'
- name: 'generator'
  content: 'TeX4ht (https://tug.org/tex4ht/)'
- name: 'viewport'
  content: 'width=device-width,initial-scale=1'
- name: 'src'
  content: '2021-07-18-hello-world.tex'
time: 1626619562
updated: 1627244699
styles:
- '2021-07-18-hello-world.css'
title: 'Hello world test'
---

   
<!--  l. 7  --><p class='indent'>   This is my test post.
</p>
    
\end{verbatim}

Although most static site generators expects Markdown, they also accept HTML files in this form.
When \texttt{staticsite} is used for the first time, it creates file with a \texttt{.published}
extension. It contains timestamp of the moment when it was used for the first time. This 
timestamp is used for the date part of the generated filename.

\section{Copy the generated files to the static site}

The \texttt{staticsite} extension can copy the generated files to places where the static site
generator expects files to process.

Let's say, that we have the following directory structure:

\begin{verbatim}
blog/
.. texposts_root/
.... first_post/
...... first_post.tex
.... second_post/
...... second_post.tex
.. html_root/
.. .make4ht
\end{verbatim}


The blog's main directory contains file \texttt{.make4ht}, and two directories: \verb|texposts_root|
and \verb|html_root|. 

The source \LaTeX\ files are stored in subdirectories of \verb|texposts_root|. We want to copy the
generated HTML files to \verb|html_root| automatically. The \texttt{staticsite} extension can be 
configured to do that using the \texttt{.make4ht} configuration file. This file is meant for passing
of shared configuration to \makeht, like in this case, where we want to specify to copy all generated
files to the \verb|html_root| directory.

The basic format of the \texttt{.make4ht} file necessary for the \texttt{staticsite} extension can look like this:

\begin{verbatim}

filter_settings "staticsite" {
  site_root =  "path/to/blogging_engine/html_dir"
  header = {
     layout="post",
  },
}

if mode=="publish" then
  Make:enable_extension "staticsite"
  Make:htlatex {}
  Make:htlatex {}
end
\end{verbatim}


The \verb|filter_settings| function passes table with settings for the extension. The \verb|site_root| field specifies path
to the generated files directory. It can be specified in the relative form, as in the example. Two \verb|..| are necessary,
as the output directory is placed two levels up in the directory hierarchy.

We also specify the build sequence for the site generation. If we pass the \verb|--mode publish| option to \makeht, the
\texttt{staticsite} extension will be enabled, and \LaTeX\ will be executed twice. This is important, because the 
contents of \verb|\title| and \verb|\author| commands are available only in the second \LaTeX\ run. They are then available
in the YAML header.

You can now execute the following command in the \verb|texposts_root/first_post| directory:

\begin{verbatim}
make4ht -m publish first_post.tex
\end{verbatim}


This will load the \texttt{staticsite} automatically, thanks to our \texttt{.make4ht} file, so it is not necessary to enable it 
on the command line. The generated HTML and CSS files will be placed in the \verb|html_root| directory.

We will take a look at how to use this setup together with Jekyll to create a simple blog in the next post.

The \href{https://github.com/michal-h21/testblog/blob/main/.make4ht}{.make4ht}
file provided in this blog repository allso adds a new function that will write 
a \verb|<input>.pubslished| file. This file is used by by the \texttt{staticsite} extension to find the original 
date of the publication of a post. You should add the \texttt{pubslished} file
to your source repository, so the correct date of the publication is used in
next updates of the site.



\section{Automatic compilation of the changed \LaTeX\ files}

Instead of compiling the documents manually after each change, 
you can automate the build process using the 
\href{https://github.com/michal-h21/siterebuild}{siterebuild} script.

This tool finds all \TeX\ files in your document directory tree for changes.
It lists only the changed files. This is especially important 
if your blog grows up, as it would be wastefull to compile all source files 
on each update.


\subsection{\texttt{rebuild.sh} Script Functionality}

I've provided a shell script named
\href{https://github.com/michal-h21/testblog/blob/main/texposts/rebuild.sh}{rebuild.sh},
included in the \TeX\ files root directory. It uses \texttt{siterebuild} and compiles the changed \TeX\ files automatically. 

It first checks for the availability of the \texttt{siterebuild} command, using
either a system-wide installation or a local version from the repository. The
script then executes \texttt{siterebuild} with debug logging to identify
modified \LaTeX\ files that require recompilation.

Using a custom output format (\verb|%dir@%file|), it extracts both the
directory path and filename for each changed document. For each detected file,
the script navigates to the corresponding directory and checks for a
\texttt{Makefile}. If present, it uses the existing build system; otherwise, it
directly invokes \texttt{make4ht} with publishing options to generate HTML
output. 

If you want to change options for \texttt{make4ht} used for the compilation, you can edit the \texttt{rebuild.sh} file. 
If you want to change options only for one file, you can create a Makefile in the directory of that file and put commands 
necessary for the compilation in that Makefile.


\end{document}
