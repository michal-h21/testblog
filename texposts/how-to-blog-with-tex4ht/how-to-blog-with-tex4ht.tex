\documentclass{article}
\usepackage{hyperref}
\title{How to Blog With TeX4ht}
\author{Michal Hoftich}
\def\makeht{\texttt{make4ht}}
\begin{document}
\maketitle

This post is part of a series on how to set up \href{https://tug.org/tex4ht/}{TeX4ht}, the \LaTeX\ to XML converter, for use with Static Site Generators. In this article, we'll discuss how to configure it to produce suitable HTML.

\tableofcontents

\section{Static site extension for make4ht}

The conversion process used by \TeX4ht is quite complex. It requires compiling a \LaTeX\ file to a DVI file with special instructions inserted by the \texttt{tex4ht.sty} package. This DVI file is then processed by the \texttt{tex4ht} command, which produces HTML or XML files, along with instructions for the final command, \texttt{t4ht}, which generates CSS files and images.

Traditionally, this process was handled by the \texttt{htlatex} script, but it had many weaknesses. The currently recommended build tool is \makeht. You can find some details about the differences between \texttt{htlatex} and \makeht\ in the \href{https://www.kodymirus.cz/make4ht/make4ht-doc.html#difference-of-texttt-makeht}{\makeht\ documentation}.

Among the features provided by \makeht\ are Lua build files, post-processing filters, and extensions. We can use these features to transform HTML files produced by \TeX4ht into the format required by static site generators.

Filters can clean up the generated files and fix common issues that are difficult to address at the \TeX\ level. They can be applied either from Lua build files or using \makeht\ extensions.

\makeht\ provides an extension that specifically supports static site generators. Let's demonstrate its usage with a simple example:

\begin{verbatim}
\documentclass{article}
\begin{document}
\title{Hello world test}
\author{Michal}
\maketitle

This is my test post.
\end{document}
\end{verbatim}

You can use the following command to generate a file suitable for static site generators:

\begin{verbatim}
make4ht -f html5+staticsite filename.tex
\end{verbatim}

By default, the \texttt{staticsite} extension produces a file named as \verb|YYYY-MM-DD-<filename>|, so this example might be named \texttt{2021-07-25-filename.html}. It's not an ordinary HTML file, but it contains a YAML header with document metadata:

\begin{verbatim}
---
meta:
- charset: 'utf-8'
- name: 'generator'
  content: 'TeX4ht (https://tug.org/tex4ht/)'
- name: 'viewport'
  content: 'width=device-width,initial-scale=1'
- name: 'src'
  content: '2021-07-18-hello-world.tex'
time: 1626619562
updated: 1627244699
styles:
- '2021-07-18-hello-world.css'
title: 'Hello world test'
---
   
<p class='indent'>   This is my test post.
</p>
    
\end{verbatim}

Although most static site generators expect Markdown, they also accept HTML files in this format. When \texttt{staticsite} is used for the first time, it creates a file with a \texttt{.published} extension. This file contains a timestamp of the moment it was first used. This timestamp is then used for the date part of the generated filename.

\section{Copy the generated files to the static site}

The \texttt{staticsite} extension can copy the generated files to the locations where the static site generator expects to find files to process.

Let's say we have the following directory structure, suitable for the \href{https://jekyllrb.com/}{Jekyll} static site generator:

\begin{verbatim}
blog/
.. texposts_root/
.... first_post/
...... first_post.tex
.... second_post/
...... second_post.tex
.. docs/_posts/
.. .make4ht
\end{verbatim}

The blog's main directory contains the file \texttt{.make4ht}, and two
directories: \verb|texposts_root| and \verb|docs/_posts|. Jekyll has built-in
support for blogs. It uses all HTML documents contained within the \verb|_posts|
subdirectory. We'll then use the \texttt{docs} directory as the source directory for
GitHub Pages.

The source \LaTeX\ files are stored in subdirectories of \verb|texposts_root|.
We want to automatically copy the generated HTML files to \verb|docs/_posts/|.
The \texttt{staticsite} extension can be configured to do this using the
\texttt{.make4ht} configuration file. This file is used to pass shared
configuration to \makeht, such as specifying that all generated files should be
copied to the \verb|docs/_posts/| directory.

The basic format of the \texttt{.make4ht} file necessary for the \texttt{staticsite} extension can look like this:

\begin{verbatim}
filter_settings "staticsite" {
  site_root = "../../docs/_posts/"
  header = {
    layout="post",
  },
}

if mode=="publish" then
  Make:enable_extension "staticsite"
  Make:htlatex {}
  Make:htlatex {}
end
\end{verbatim}

The \verb|filter_settings| function passes a table with settings for the
extension. The \verb|site_root| field specifies the path to the directory for
the generated files. It can be specified in a relative form, as in this
example. Two \verb|..| are necessary because the output directory is located
two levels up in the directory hierarchy from the directory of the compiled \TeX\ file.

We also specify the build sequence for site generation. If we pass the
\verb|--mode publish| option to \makeht, the \texttt{staticsite} extension will
be enabled, and \LaTeX\ will be executed twice. This is important because the
contents of the \verb|\title| and \verb|\author| commands are only available in
the second \LaTeX\ run. They are then included in the YAML header.

You can now execute the following command in the \verb|texposts_root/first_post| directory:

\begin{verbatim}
make4ht -m publish first_post.tex
\end{verbatim}

This will automatically load the \texttt{staticsite} extension, thanks to our \texttt{.make4ht} file, so it's not necessary to enable it on the command line. The generated HTML and CSS files will be placed in the \verb|docs/_posts/| directory.

In the next post, we will look at how to use this setup with Jekyll to create a simple blog.

The \href{https://github.com/michal-h21/testblog/blob/main/.make4ht}{.make4ht} file provided in this blog repository also adds a new function that writes a \verb|<input>.published| file. This file is used by the \texttt{staticsite} extension to find the original publication date of a post. You should add the \texttt{published} file to your source repository so the correct date is used in future updates of the site.

\section{Automatic compilation of changed \LaTeX\ files}

Instead of compiling documents manually after each change, you can automate the build process using the \href{https://github.com/michal-h21/siterebuild}{siterebuild} script.

This tool checks all \TeX\ files in your document directory tree for changes and lists only the modified files. This is especially important as your blog grows, as it would be wasteful to compile all source files on every update.

\subsection{\texttt{rebuild.sh} Script Functionality}

I've provided a shell script named \href{https://github.com/michal-h21/testblog/blob/main/texposts/rebuild.sh}{rebuild.sh}, which is included in the \TeX\ files root directory. It uses \texttt{siterebuild} to automatically compile the changed \TeX\ files:

\begin{verbatim}
#!/bin/sh
if ! command -v siterebuild &> /dev/null
then
SITEREBUILD=../siterebuild/siterebuild
else
SITEREBUILD=siterebuild
fi

export TEXINPUTS=.:/root/texmf//: 

$SITEREBUILD -l debug
# we use the custom output format for siterebuild, to be able to easily extract directory and filename in the later steps
for i in `$SITEREBUILD -o %dir@%file`
do
  texdir=`echo $i | cut -d@ -f1 -`
  texfile=`echo $i | cut -d@ -f2 -`
  # either execute Makefile, or run make4ht directly
  cd "$texdir"
  if test -f Makefile; then
    make
  else
    TEXINPUTS=.:/root/texmf//: make4ht -a debug -m publish -l "$texfile"
  fi
  cd ..
done
\end{verbatim}

It first checks for the availability of the \texttt{siterebuild} command, using
either a system-wide installation or a local version from the repository. The
script then executes \texttt{siterebuild} with debug logging to identify
modified \LaTeX\ files that require recompilation.

Using a custom output format (\verb|%dir@%file|), it extracts both the
directory path and filename for each changed document. For each detected file,
the script navigates to the corresponding directory and checks for a
\texttt{Makefile}. If one is present, it uses the existing build system;
otherwise, it directly invokes \texttt{make4ht} with publishing options to
generate the HTML output.

If you want to change the options for \texttt{make4ht} used for the
compilation, you can edit the \texttt{rebuild.sh} file. If you want to change
options for only one file, you can create a \texttt{Makefile} in that file's
directory and put the necessary compilation commands in that \texttt{Makefile}.


\end{document}
