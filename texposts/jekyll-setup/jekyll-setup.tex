\documentclass{article}
\title{Jekyll Setup}
\author{Michal Hoftich}
\def\makeht{\texttt{make4ht}}
\usepackage{hyperref}
\usepackage{lipsum}
\begin{document}
\maketitle

I've showed how to use \makeht\ \texttt{staticsite} extension
in the \href{/testblog/2021/07/30/how-to-blog-with-tex4ht.html}
{previous article}. In this article, I will show how to 
setup it together with \href{https://jekyllrb.com/}{Jekyll}
to create a simple blog.

\tableofcontents

\section{Create new Jekyll site}

We assume that the HTML files generated from \LaTeX\ using \makeht\ 
are stored in a directory called \texttt{html\_posts}. 
You can initialize a new Jekyll site in this directory using the following commands:

\begin{verbatim}
$ cd html_posts
$ jekyll new .
\end{verbatim}

This command creates a new Jekyll site in the current directory, without overwriting existing files.

\section{Configure Jekyll}

To improve how Jekyll handles the content, update the \texttt{\_config.yml} file 
with the following options:

\begin{verbatim}
show_excerpts: true
excerpt_separator: "</p>"
\end{verbatim}

This enables post excerpts and specifies that the excerpt ends at the first closing paragraph tag.

\subsection{Customizing the layout}

You can modify the default HTML layout by editing files in the \texttt{\_includes} and \texttt{\_layouts} folders. 
For example, create a custom layout that uses the styles and metadata provided by your \LaTeX\ output.

Edit the file:

\begin{verbatim}
docs/_includes/head.html
\end{verbatim}

\subsection{Insert document stylesheet}

In the \texttt{head.html} file, make sure to include both the default Jekyll styles
and any additional styles generated by \makeht. Here's an example of a complete \texttt{<head>} section:

\ifdefined\HCode\verb||\fi
\begin{verbatim}
<head>
  <meta charset="utf-8">
  <meta http-equiv="X-UA-Compatible" content="IE=edge">
  <meta name="viewport" content="width=device-width, initial-scale=1">
  
  <link rel="stylesheet" href="{{ "/assets/main.css" | relative_url }}" />
  
  <link rel="stylesheet" href="/css/{{style}}" />
  
  
  
    
  
</head>
\end{verbatim}
\ifdefined\HCode\verb||\fi

The loop over \texttt{page.styles} allows dynamic inclusion of CSS files 
exported during the \LaTeX\ to HTML conversion.

\section{\LaTeX\ compilation}

To generate clean HTML that is compatible with Jekyll, 
use the \texttt{staticsite} extension of \makeht. I've described
how to use it in the \href{testblog/2021/07/30/how-to-blog-with-tex4ht.html}{previous post}.


\section{Conclusion}

Using \makeht\ together with Jekyll allows you to write blog posts in \LaTeX\ 
while benefiting from static site generation. 

\end{document}
