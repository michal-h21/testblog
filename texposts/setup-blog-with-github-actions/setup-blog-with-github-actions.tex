\documentclass{article}
\usepackage{hyperref}
\title{Setup Blog with GitHub Actions}
\author{Michal Hoftich}
\def\makeht{\texttt{make4ht}}
\begin{document}
\maketitle

This article describes how to configure GitHub Actions to automatically compile \LaTeX\ sources
for a static blog built with tools like \makeht{} and Jekyll. It covers file tracking,
automatic build triggering, and committing generated files back to the repository.

\tableofcontents

\section{Problem: Git and Modification Times}

Git does not preserve file modification times when cloning or checking out repositories.  
However, tools like \makeht{} or other build systems may rely on these timestamps to detect changed files.  
To compile only updated \LaTeX\ sources, we need to **restore original modification times** 
from the commit history.

\subsection{Restoring Timestamps with GitHub Actions}

The 
\href{https://github.com/chetan/git-restore-mtime-action}{\texttt{git-restore-mtime-action}} 
can be used to recover accurate modification times for files in a GitHub workflow.

To function correctly, it requires the full commit history, not just the latest commit.
You must ensure the following step is included in your GitHub Actions workflow:

\begin{verbatim}
- uses: actions/checkout@v2
  with:
    fetch-depth: 0
\end{verbatim}

This fetches the complete Git history so the action can determine modification times correctly.

\section{Compiling \LaTeX\ on GitHub}

It is possible to run the full \LaTeX{} to HTML compilation process **directly on GitHub**, 
without needing to generate files locally.

\subsection{Using a Marker File to Trigger Compilation}

To control whether a file should be rebuilt, we use a marker file named after the source file 
with the extension \texttt{.published}. It stores the timestamp of the last successful publication.

For example, for a file named \texttt{tex\_filename.tex}, create a corresponding marker file:

\begin{verbatim}
$ date +%s > tex_filename.published
\end{verbatim}

This command stores the current UNIX timestamp in the marker file. The build script 
on the GitHub server can then compare this value with the file's modification time 
to decide whether recompilation is needed.

\section{Committing Generated Files Back to GitHub}

If your build process generates HTML or other output files that should be committed 
back to the repository (e.g., for GitHub Pages), you can automate the commit using:

\href{https://github.com/stefanzweifel/git-auto-commit-action}{\texttt{git-auto-commit-action}}

It allows you to commit changes created during the GitHub Action run.

A typical workflow step might look like this:

\begin{verbatim}
- uses: stefanzweifel/git-auto-commit-action@v5
  with:
    commit_message: "Update generated HTML files"
    branch: main
\end{verbatim}

This commits and pushes the generated files to the \texttt{main} branch automatically, 
making them available to GitHub Pages or other deployment mechanisms.

\section{Conclusion}

With a combination of restored modification times, marker files, and automated commits, 
you can create a seamless workflow for compiling and deploying your \LaTeX{}-authored blog 
entirely on GitHub. This reduces the need for manual steps and makes continuous publishing possible.

\end{document}
